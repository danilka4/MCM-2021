
\documentclass[2126824.tex]{subfiles}

\begin{document}
We now seek to develop a model capable of identifying whether user submitted photos constitute a Positive ID, with the objective of using this model being to identify sightings amongst the Unverified data set that warrant further investigation. We assume there are three potential reasons a report that is likely credible would end up in the Unverified Category other than a simple case of clerical or other human error: poor image quality, discrepancies elsewhere in the report, or the content of the report. In cases of poor image quality being the leading factor in the misclassification, it is possible the model will be able to distinguish certain features the human eye misses in a low resolution photograph. In cases where a video was submitted with the report, the model can also be applied on individual frames, which may lead to a higher rate of accuracy than human analysis. 
Discrepancies elsewhere in reports that could to misclassification could include empty fields and obviously incorrect dates.  Empty fields could lead to discarding simply because they don’t provide all the required data. Reports with incorrect dates could perhaps lead reviewers to instantly discard the report, although in the case of incorrect dates, it seems unlikely that someone would unintentionally type in anything other than an easy to interpret typo, such typing ‘1/2/020’ instead of ‘1/2/2020’, or 3/62/020’ – however, there are cases where shifting a number leads to added ambigouity: ‘32/6/2020’ could lead a reviewer guessing whether the 3 was a misclick (intended text: ‘2/6/2020’), or if the 2 was intended to after the first forward slash. Some incorrect dates, such as “515”, appear too off to have been unintentional. However, we assume in the vast majority of cases with incorrect dates, a typo is only main in one of date fields, Detection Date or Submission date, not both – and in some instances this could lead to an easy identification of what the reporter intended to write. In any case, we assume this will be a small portion of any misclassified reports. 
In the case of content leading to misclassification, we mean a case where the text included in the notes field led to the reviewer discarding the report. We do not know the order in which the reviewers check the data fields, but we assume that if the notes field is read before looking at the image, certain text contents could lead to report ‘not bothering’ to check the image. Fictional examples of these could include:
(1)	“I don’t know where in WA my friend was when she took this photo but I’m submitting it anyways.”
(2)	“I think it’s a bee.”
(3)	“We gave the coordinates of the nearest town.”
While sentences (1) and (2) could likely cause someone to discard a report, we can still make use of the valid data provided if the report is indeed positive. In sentence (2), the reporter said something that made the report seem pointless. 
\end{document}