\documentclass{article}
\usepackage[utf8]{inputenc}

\title{MMC}
\author{victor19 }
\date{February 2021}

\begin{document}

\maketitle

\section{Introduction}



%Distribution of Data by ID Type table
\begin{center}
 \begin{tabular}{||c c c c c||} 
 \hline
 \multicolumn{5}{|c|}{Distribution of Data by ID Type} \\
 \hline
 Category & Positive ID & Negative ID & Unverified & Unprocessed  \\ [0.5ex] 
 \hline
 Number of Reports & 14 & 2069 & 2342 & 15 \\ 
 \hline
\% of Reports & .3182\% & 47.0227\% & 53.2273\% & .3409\% \\

 \hline
\end{tabular}
\end{center}

%Distribution of Data by Detection Date and Category table

\begin{center}
 \begin{tabular}{||c c c c c||} 
 \hline
 \multicolumn{5}{|c|}{Distribution of Data by Detection Date and ID Type} \\
 \hline
 Category & Positive ID & Negative ID & Unverified & Unprocessed  \\ [0.5ex] 
 \hline
 \# of Reports Before 1980 & 0 (0\%) & 1 (.0483\%) & 10 (.4270\%) & 0 (0\%) \\ 
 \hline
\# of Reports Before 2000 & 0 (0\%) & 1 (.0483\%) & 11 (.4697\%) & 0 (0\%) \\
\hline
\# of Reports Before 2010 & 0 (0\%) & 1 (.0483\%) & 12 (.5124\%) & 0 (0\%) \\
\hline
\# of Reports Before Sep, 2019 & 0 (0\%) & 35 (1.6916\%) & 131 (5.5935\%) & 2 (13.3333\%) \\
\hline
\end{tabular}
\end{center}

\section{"The Situation" (should go in summary? idk)}
Vespa Mandarina, commonly referred to as Asian Giant Hornets, sparrow wasps, and “killer hornets”, are a species of large hornet most commonly found in Japan [penn state source]. With recent confirmed sightings on Vancouver Island and in Washington States, there has been growing public alarm about a larger scale invasion of the western North America region. This alarm is indeed justified, as Asian Hornets have demonstrated themselves to be dominating invasive species, under certain circumstances. One of the key attributes that allows hornets to be so easily invasive is their adaptability to new environments due to their eusocial behavior –– however, in new environments hornet species will often lose their eusocial behavior and live as ‘rogue’ wasps. Without the power of eusocial behavior, Asian hornets lose their ability to reproduce at high rates, thus decreasing their overall population – furthermore, they lose their ability to swarm beehives, although they will still pick off individual bees. According to [French study], only 9 out of 34 introductions of the species to new environments have created an invasive species issue – but it is important to note that most of these introductions have been in Europe, and whether the species becomes invasive is depend on environment in which the hornets find themselves and the susceptibility of native species to lose resources or members to the hornets. Of particular susceptibility to the hornets are honeybees, causing a rising concern amongst beekeepers and the creation of a hive attack hotline in Washington State. While bees in Japan have had ample opportunity to adapt, going as far as to even develop specific tactical response to hornets (French source), this response is not present in honeybees cultivated in America, who are further rendered defenseless by the Asian Hornet’s immunity to bee stings. 
The dangers of an expanding Asian Hornet population in North America include economic impact to beekeepers, who may have to invest in hornet management or relocate (‘assesing the eco.’), human danger from the highly venomous stings, environmental impact of their presence and the subsequent elimination of invasive species, and further expansions into other areas. With regard to further expansion, it should be noted that V. mandarina has a tendency to relocate by hibernating or building nests in shipping containers (‘assesing the eco.), of which there are plenty in region, with 75 shipping ports in Washington State alone. This could lead to reintroduction of the species even after elimination, and could also result in spread to other areas across the world. Furthermore, while there is no evidence of Asian hornets in eastern America, [‘assesing the eco.’] warns that areas across North America (as well as parts of South America, central Africa, eastern Australia, and New Zealand) have highly suitable climates for the Asian Hornets to thrive. 


\end{document}